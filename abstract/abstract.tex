\addcontentsline{toc}{chapter}{Abstracto}

\begin{abstract}

Existen muchas teorías computacionales desarrolladas para mejorar el desempeño en clasificación fonética obtenido desde flujos lingüísticos auditivos. Sin embargo, no se le ha prestado mucha atención a las características neuro-fisiológicas halladas en el tejido cortical. Nos enfocamos en el hecho de que los humanos, así como otros animales tienen la capacidad de clasificar de manera robusta unidades lingüísticas básicas--como fonemas--extraídas desde flujos acústicos complejos en los datos producidos por el habla. En este trabajo, introducimos un enfoque computacional biológicamente inspirado y completamente no supervisado que incorpora propiedades corticales neuro-fisiológicas y anatómicas claves. La capacidad de abstracción de características de este enfoque ha mostrado atributos de generalización e invarianza fonética prometedores. Este modelo mejora el desempeño en clasificación fonética de la técnica supervisada \gls{svm} para tareas de clasificación de palabras monosilábicas, bisilábicas y trisilábicas en presencia de ruido blanco, reverberación y variaciones de tono. De esta manera, el modelo computacional presentado en este trabajo supera claramente sofisticadas representaciones espectro-temporales de múltiple resolución de los datos lingüísticos fonéticos de entrada.

\end{abstract}
